
\section{Replicated document}
\label{sec:structure}

To provide eventually consistent replicas, \CRATE uses a conflict-free
replicated data type for sequences. Such sequence type avoids the difficult and
error-prone task of solving conflicts by the mean of commutative
operations. Thus, the sequence type provides two basic operations: the insertion
and removal of an element. If two users perform concurrently some operations at
an identical position in the sequence, the latter will still converge to an
equivalent state.

Commutativity comes at the price of additional metadata attached to each
element of the sequence. These metadata referred to as identifier are unique,
immutable, and of variable size at generation. When a character is inserted, the
sequence type generates an identifier which is a list
$[\ell_1.\ell_2\ldots \ell_k]$ where $k$ is the identifier depth and Element
$\ell_i$ comprises a path $p_i$, a globally unique site identifier $s_i$, a
monotonic local counter $c_i$.

A lexicographic total order amongst these identifiers allows retrieving the
sequence:
\begin{align*}
  \ell_i < \ell_j \iff & (p_i < p_j) \vee \\
                       & ((p_i = p_j) \wedge (s_i<s_j)) \vee \\
                       & ((p_i = p_j) \wedge (s_i = s_j) \wedge (c_i < c_j)) \\
  \ell_i = \ell_j \iff & \neg (\ell_i < \ell_j) \wedge \neg (\ell_j < \ell_i) \\
  i_i < i_j \iff & \exists (m > 0)(\forall n < m) (\ell^i_n = \ell^j_n) \wedge (\ell^i_m < \ell^j_m) \\
  i_i = i_j \iff & (\forall m) \ell^i_n = \ell^j_n
\end{align*}

The first two equivalences define an ordering between elements of the list
composing the identifiers. The last two equivalences use these to define a
global total order amongst identifiers, hence, ordering characters of the
document. While the third equivalence is mostly used to locate the proper
inserting position, the fourth equivalence is necessary for removals.  Paths of
identifiers constitute the most discriminant part of identifiers, hence the most
important part. The rest of this section focuses on these.

\CRATE uses \LSEQ to generate the paths of identifiers. The paths are series of
integers $[p_1.p_2\ldots p_k]$ where each path is chosen among a set twice as
large as its preceiding path. For instance, if $p_1$ is chosen among
$\{0..2^8\}$ then $p_2$ is chosen among $\{0..2^9\}$ etc.  When the document
grows, the depth of identifiers is expected to grow. Therefore, increasing the
size of sets over depths decreases the growth in depth. In turns, \LSEQ
allocates identifiers sub-linearly upper-bounded compared to the number of
insertions in the sequence.

%%% Local Variables:
%%% mode: latex
%%% TeX-master: "../paper"
%%% End:
