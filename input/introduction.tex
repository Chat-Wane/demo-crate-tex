
\section{Introduction}
\label{sec:introduction}

Real-time collaborative editors allow authors to simultaneously write a shared
document. While being undoubtedly useful, most of current tools available belong
to the Cloud (\REF). With centralization come privacy and scalability issues. To
reconcile with the Internet working distributed without central authorities,
editors start to get rid of such centrality point, which in turns solves privacy
issues. Still scalability problem remains, and usability formerly provided by
the Cloud is gone.

\CRATE is a distributed and decentralized \textsc{c}ollabo\textsc{rat}ive
\textsc{e}ditor providing scalable real-time editing capabilities directly
within modern web browsers. As such, users of heterogeneous devices are able to
access to an editing session as simply as clicking a URL (\emph{Uniform Resource
  Locator}).

To provide availability and responsiveness of documents, \CRATE uses the
optimistic replication scheme (\REF). Thus, each editor replicates the document
locally and directly performs operations on it. Changes are spread to all
editors. When the same set of operations are applied by editors, the replicates
converge to an equivalent state allowing users to read the same document.  This
property is called \emph{strong eventual consistency} (\REF). A plethora of
approaches provide such consistency criteria. \CRATE uses a data structure that
avoids conflicting operations by design (\REF). It relieves users from the
burden of solving conflicts at the cost of additional metadata the size of
which is kept \TODO{reasonable} thanks to \LSEQ. 

To spread changes made on documents to the editing session, \CRATE firstly
builds a network of browsers which is secondly used to disseminate messages in a
scalable way (\REF). Such feature is only allowed by the recent introduction of
WebRTC\footnote{\url{http://www.webrtc.org}} which allows browser-to-browser
communication.

