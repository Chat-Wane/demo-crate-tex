
\section{Introduction}
\label{sec:introduction}

Real-time collaborative editors allow authors to simultaneously write shared
documents~\cite{greenberg1994real}. Trending editors such as Google Docs rely on
central servers. They raise privacy issues since service providers take
advantage of their mediating position to observe all users and documents. They
also raise scalability issues, for they require a server to support an editing
session, so handling many editing sessions requires many resources. Also
managing large editing sessions puts a lot of pressure on the server which may
become overloaded.

%% managing many participants in one editing session is costly and may overload the
%% server.

\CRATE is a distributed and decentralized \textsc{c}ollabo\textsc{rat}ive
\textsc{e}ditor providing scalable real-time editing capabilities directly
within web browsers. Compared to state-of-the-art, \CRATE is the first real-time
editor that only requires browsers to support collaborative editing and to
transparently handle from small to large groups of users.

Suppose massive online lecture platforms allow students to share their
notes. Many lectures run in parallel involving various number of students, i.e.,
from few to thousands. Also, even during the editing session the audience is subject
to significant changes in size, going from thousands to hundreds. Finally, the
resulting document size is hardly predictable, for it depends of the number and
zealous of involved students.
%% even during one editing session, the number of participants can change
%% significantly, from thousands to few hundreds. The size of document
%% produced by users is also hard to predict: it can vary from few pages
%% to hundreds of pages. 

\CRATE handles these situations by combining different algorithms to achieve an
acceptable trade-off between space, time and communication complexities. In
laboratory, we observed this trade-off on configurations involving up till 600
browsers. In this paper, we describe the usage, the architecture and the
properties of \CRATE. Also we propose a live experiment to WWW2016 participants
aiming to confirm laboratory experiments about \CRATE.

Section~\ref{sec:usage} depicts the usage of \CRATE.
Section~\ref{sec:architecture} details the overall architecture of \CRATE along
with its basic functioning. Section~\ref{sec:network} shows the way to build
decentralized networks directly in browsers.  Section~\ref{sec:structure}
describes the distributed data structure that represents the document.
Section~\ref{sec:live} shows our laboratory result and describes the live
experiment setup. Section~\ref{sec:conclusion} concludes.


%%% Local Variables:
%%% mode: latex
%%% TeX-master: "../paper"
%%% End:
