
\section{Introduction}
\label{sec:introduction}

Real-time collaborative editors allow authors to simultaneously write shared
documents. While being undoubtedly useful, most of current tools available
belong to the Cloud (\REF). With centralization come privacy and scalability
issues. To reconcile with the Internet working distributed without central
authorities, editors start to get rid of such centrality point, which in turns
solves privacy issues. Still scalability problem remains, and usability formerly
provided by the Cloud is gone.

\CRATE is a distributed and decentralized \textsc{c}ollabo\textsc{rat}ive
\textsc{e}ditor providing scalable real-time editing capabilities directly
within modern web browsers. As such, users of heterogeneous devices are able to
access to an editing session as simply as clicking a URL (\emph{Uniform Resource
  Locator}).

To provide availability and responsiveness of documents, \CRATE uses the
optimistic replication scheme (\REF). Thus, each editor replicates the document
locally and directly performs operations on it. Changes are spread to all
editors. When the same set of operations are applied by editors, the replicates
converge to an equivalent state allowing users to read the same document.  This
property is called \emph{strong eventual consistency} (\REF). A plethora of
approaches provide such consistency criteria. \CRATE uses a data structure that
avoids conflicting operations by design (\REF). It relieves users from the
burden of solving conflicts at the cost of additional metadata the size of
which is kept \TODO{reasonable} thanks to \LSEQ. 

To spread changes made on documents to the editing session, \CRATE firstly
builds a network of browsers which is secondly used to disseminate messages in a
scalable manner (\REF). Such feature is only allowed by the recent introduction
of WebRTC\footnote{\url{http://www.webrtc.org}} as a browser-to-browser
communication protocol. To build the network of browsers, \CRATE uses \SPRAY: a
random peer sampling protocol (\REF) which provides each editor with a
neighborhood table to communicate with. The size of the latter adapts to the
editing session size automatically and without any global knowledge. The
dissemination protocol built on top of it directly beneficits from this
property. Thus, traffic follows the needs of the editing session.

This demo paper
\begin{inparaenum}[(i)]
\item describes the aforementionned components that build \CRATE;
\item describes the setup of our live demo where people would get the
  opportunity to join an editing session and start the redaction of a document
  together. We hope to retrieve a meaningful document along with interesting
  measurements.
\end{inparaenum}

The rest of this paper is organized as follow: Section~\ref{sec:architecture}
details the overall architecture of \CRATE along with its basic
functionning. Section~\ref{sec:structure} describes the distributed data
structure that represents the document. Section~\ref{sec:network} details the
network building operation. Section~\ref{sec:live} states the live experiment
setup we propose. Section~\ref{sec:conclusion} concludes.


