
\section{Introduction}
\label{sec:introduction}

Real-time collaborative editors allow authors to simultaneously write
shared documents~\cite{greenberg1994real}. Trending editors such as
Google Docs rely on central servers. They raise privacy issues since
service providers take advantage of their mediating position to
observe all users and documents. They also raise scalability issues: a
server is required to support one editing session, so handling many
editing sessions require many ressources. Also managing many
participants in one editing session is costly and may overload the server.

\CRATE is a distributed and decentralized
\textsc{c}ollabo\textsc{rat}ive \textsc{e}ditor providing scalable
real-time editing capabilities directly within existing web
browsers. Compared to state of art, \CRATE is the first real-time
editor that only require browsers to support collaborative editing and
handle transparently small groups to large groups of
users. 
Suppose a massive online lecture platfrom allows participants to share
notes about the lecture. Many lectures can run in parallel with
various number of participants ie. from few to thousands. Even during
one editing session, the number of participants can change
significantly, from thousands to few hundreds. The size of document
produced by users is also hard to predict: it can vary from few pages
to hundreds of pages. 

\CRATE handles these situations by combining different algorithms to
achieve an acceptable trade-off between space, time and communication
complexities. We observed this trade-off \CRATE in lab on configurations up
to 600 browsers. In this paper:
\begin{inparaenum}[(i)]
\item we describe the architecture and the properties of the \CRATE
  decentralized real-time editor;
\item we describe its usage through a walkthrough example. We propose
  a live experience to WWW2016 participants.
\end{inparaenum}

Section~\ref{sec:architecture} details the overall architecture of \CRATE along
with its basic functioning. Section~\ref{sec:structure} describes the
distributed data structure that represents the
document. Section~\ref{sec:network} details the network building
operation. Section~\ref{sec:live} states the live experiment setup we
propose. Section~\ref{sec:conclusion} concludes.


%%% Local Variables:
%%% mode: latex
%%% TeX-master: "../paper"
%%% End:
