
\section{Usage}
\label{sec:usage}

\begin{figure*}
  \centering
  \includegraphics[width=0.95\textwidth]{./img/crate.png}
  \caption{\label{img:screenshot} Screenshot of the web application containing
    two connected editors: on the left, a document is written in markdown
    language which is previewed on the right editor.}
\end{figure*}

\CRATE's video, source code and online demo are freely available at
\url{https://github.com/Chat-Wane/CRATE}.

In the online demo\footnote{\url{http://chat-wane.github.io/CRATE/}, the browser
  must support WebRTC}, a user creates a document by clicking the $\oplus$ icon
at the top of the screen (see Figure~\ref{img:screenshot}). At the time, the
document is only local and no one can read nor modify it apart from its
creator. When the user is ready to share, she clicks on the $chain$ icon and
\CRATE is crafting an editing session URL that can be sent by mail, published on
Twitter, or advertised by any other mean.

Once the collaborators open the link, it automatically connects them to the
editing session of the creator. It retrieves the shared document and they can
start the real-time collaborative editing. In turns, they are able to share the
access to the document too.

Figure~\ref{img:screenshot} shows a screenshot of the graphical user interface
of \CRATE. In this scenario, two editors are running in a same instance of the
web application. The editing session comprises two members nonetheless. The
leftmost author writes the \emph{readme} file of the Github repository about
\CRATE. The green earth indicates that the editing session is live. The blue
(spinning) circle indicates that she opens the access to her
document. Therefore, anyone can join it with the proper URL. The rightmost
editor renders the document written in markdown language. It has a green earth
indicating a connected state but contrarily to the leftmost editor, it is not
sharing the access to the document.

URLs are the means for accessing to editing sessions, hence, to
documents. Internally, when a browser opens the link it retrieves the web
application. The latter queries a mediator server with the parameters contained
within the link (see Figure~\ref{fig:spray}). The mediator server tries to
establish the very first contact between the joining browser and an editing
session sharer.  The mediator ensures solely a minimal service consisting in the
editing session accessibility.

Yet, an URL does not always grant the access to the live editing session, for
there must be at least one sharer among the editors. Also, there may exist
multiple URLs leading to a same editing session using different joining
parameters. As we observe in Figure~\ref{img:screenshot}, \CRATE enables
hyperlinks. By the same, it allows surfing between editing sessions like any
user would do on Wikipedia, enhanced with real-time capabilities.

%%% Local Variables:
%%% mode: latex
%%% TeX-master: "../paper"
%%% End:
