
\section{Architecture}
\label{sec:architecture}

\begin{figure}
  \centering
  \begin{tikzpicture}[scale=1]

\newcommand\X{19pt}
\newcommand\Y{20pt}

\newcommand\LIGHTERGRAY{base3}
\newcommand\LIGHTGRAY{base2}
\newcommand\MEDIUMGRAY{yellow}

\scriptsize
%% communication
\draw[rounded corners=2mm, color=\MEDIUMGRAY, fill=\LIGHTERGRAY](0pt, 0pt)+(-4*\X,-\Y)rectangle+(4*\X,\Y);


\draw[fill=\LIGHTERGRAY](-2*\X, -0.25*\Y)
node{broadcast}+(-1*\X,-0.5*\Y)rectangle+(1*\X,0.5*\Y);
\draw[fill=\LIGHTERGRAY, very thick]( 0.7*\X, 0.25*\Y)
node[align=center]{membership\\\SPRAY\footnote{Nédelec et al. (2015) Spray: an Adaptive Random Peer Sampling Protocol}}+(-1.2*\X,-0.5*\Y)rectangle+(1.2*\X,0.5*\Y);

\draw[<-](-0.5*\X, 0.25*\Y)--(-1*\X, -0.25*\Y);
%% \draw[<-](0.85*\X, 0.25*\Y)--(1.25*\X, -0.25*\Y);

%% causality
\draw[rounded corners=2mm, color=\MEDIUMGRAY, fill=\LIGHTGRAY](0pt, -2*\Y)+(-4*\X,-\Y)rectangle+(4*\X,\Y);


\draw[fill=\LIGHTGRAY](-2*\X, -2*\Y)
node[align=center]{version vector\\with\\exceptions\footnote{Malkhi et al. (2007) Concise version vectors in WinFS}}
+(-1.5*\X,-0.6*\Y)rectangle+(1.5*\X,0.6*\Y);
\tiny
\draw[->, thick, color=red](-1.5*\X, -0.75*\Y) -- node[anchor=west]{\ receive}
(-1.5*\X, -1.4*\Y);
\draw[<-, thick, color=blue](-2.5*\X, -0.75*\Y) -- node[anchor=east]{send \ }
(-2.5*\X, -1.4*\Y);
\scriptsize
%% \draw[<->]( 2*\X, -0.75*\Y)--( 1*\X, -2.5*\Y);

%% sequence structure
\draw[rounded corners=2mm, color=\MEDIUMGRAY, fill=\LIGHTERGRAY](0pt, -4*\Y)+(-4*\X,-\Y)rectangle+(4*\X,\Y);

%% \draw[fill=white, shading=axis,top color=\LIGHTGRAY, bottom color=white, shading angle=0](1*\X, -3*\Y)
%% node{anti-entropy}+(-0.95*\X,-0.5*\Y) rectangle +(0.95 *\X, 0.5*\Y);
\draw[fill=\LIGHTERGRAY, very thick](-2*\X, -4*\Y)
node{\LSEQ\footnote{Nédelec et al. (2013) LSEQ: an Adaptive Structure for Sequences in Distributed Collaborative Editing}}+(-0.75*\X,-0.5*\Y) rectangle +(0.75 *\X, 0.5*\Y);

%\draw[->] (0.05*\X, -2.75*\Y)--(-1*\X,-2*\Y);
%\draw[->] (0.05*\X, -3.25*\Y)--(-1.25*\X,-4*\Y);
\tiny
\draw[<-, thick, color=red] (-1.5*\X, -3.5*\Y)--node[anchor=west]{\ deliver}(-1.5*\X, -2.6*\Y);
\draw[->, thick, color=blue] (-2.5*\X, -3.5*\Y)--node[anchor=east]{decorate \ }(-2.5*\X, -2.6*\Y);
\scriptsize
%% gui
\draw[rounded corners=2mm, color=\MEDIUMGRAY, fill=\LIGHTGRAY](0pt, -6*\Y)+(-4*\X,-\Y)rectangle+(4*\X,\Y);

\draw[fill=\LIGHTGRAY](0.7*\X,-6*\Y)
node{web editor\footnote{\url{ace.c9.io}}}+(-1*\X,-0.5*\Y) rectangle +(1*\X, 0.5*\Y);

%%\draw[<->] (-2*\X, -4.5*\Y) -- (0*\X, -5.5*\Y);
\tiny
\draw[->, thick, color=red] (-1.80*\X, -4.5*\Y)--node[anchor=west]{\ notify}(-0.3*\X, -5.75*\Y);
\draw[<-, thick, color=blue] (-2.30*\X, -4.5*\Y)--node[anchor=east]{update \ }(-0.3*\X, -6.25*\Y);
\scriptsize

\draw(4*\X, -\Y)node[anchor=south east]{\textbf{communication}};
\draw(4*\X, -7*\Y)node[anchor=south east, align=right]
{\textbf{graphical}\\\textbf{user}\\\textbf{interface}};
\draw(4*\X, -5*\Y)node[anchor=south east, align=right]
{\textbf{sequence}\\\textbf{structure}};
\draw(4*\X, -3*\Y)node[anchor=south east]{\textbf{causality}};


\end{tikzpicture}
  \caption{\label{fig:architecture} Four layers architecture of \CRATE.}
\end{figure}

\CRATE is based on the optimistic replication
scheme~\cite{saito2005optimistic}. Thus, each editor replicates the
document locally and directly performs operations on it. Next, changes
are spread to all other participants. When the same set of operations
are applied by editors, the replicates converge to an equivalent state
allowing users to read the same document.  This property is called
\emph{strong eventual consistency}~\cite{bailis2013eventual}.

\CRATE uses a Conflict\--Free Replicated Data structure (CRDT) for
sequences to ensure strong eventual consistency on the
document~\cite{shapiro2011comprehensive}. Consistency is ensured at
the price of unique identifier attached to each element of the
sequence. Recently, \LSEQ~\cite{nedelec2013lseq} proposed a strategy
to bound the space complexity of these identifiers to $log(n)^2$ where
$n$ is the total size of document. This results avoids to run costly
distributed garbage-collection-like mechanism to maintain the
efficiency of the editor.

Strong consistency is ensured if every operations sent by a user is
eventually received by all participants of the editing
sessions. \CRATE builds an editing session using
\SPRAY~\cite{nedelec2015spray}, a random peer sampling
protocol~\cite{jelasity2007gossip} (RPS) built on top of WebRTC. A
peer-sampling protocol allow participants of a network to just have a
random partial view of whole network. Maintaining this partial view
random ensures the connectivity of the network. Unlike prior RPS,
\SPRAY allow to adapt the size of the partial view to the editing
session size without measuring the number of participants. Since the
propagation protocol of messages~\cite{birman1999bimodal} extensively
uses these tables, the network traffic inherits from this
scalability. Consequently, \CRATE adapts its operation to the need of
the editing session size.


The distributed and decentralized \textsc{c}ollabo\textsc{rat}ive
\textsc{e}ditor \CRATE developed in web languages only: HTML, CSS, and mostly
JavaScript. As shown in Figure~\ref{fig:architecture}, \CRATE comprises four
layers:
\begin{enumerate}[(i)]
\item [\textbf{the graphical user interface}] that renders the document to the users
  (cf. Figure~\ref{img:screenshot}). Each local update is immediatly applied to
  the document view for real-time sake. Then, the update is applied to the
  replicated data type;
\item [\textbf{the replicated data type layer}] that represents the local document
  (cf. §\ref{sec:structure}). It is in charge of providing the metadata
  necessary to order each character identically anywhere;
\item [\textbf{the causality layer}] that tracks semantic causality between operations,
  e.g., the removal of a character cannot precede its insertion;
\item [\textbf{the network layer}] that
  \begin{inparaenum}
  \item builds a network of browsers for each editing session and
  \item uses it to propagate the updates to all collaborators
    (cf. §\ref{sec:network}).
  \end{inparaenum}
\end{enumerate}



%%% Local Variables:
%%% mode: latex
%%% TeX-master: "../paper"
%%% End:
