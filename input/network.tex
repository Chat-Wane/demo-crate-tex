
\section{Network of browsers}
\label{sec:network}

\CRATE uses \SPRAY to build a network of browsers thanks to WebRTC. The latter
is a recent technology that allows establishing browser-to-browser channels of
communication. Its connection establishment protocol imposes a round-trip of
messages. Thus, if an editor $e_1$ wants to connect to another editor $e_2$, it
needs to send a message to $e_2$ and to acknowledge its response. While the
first of such connection must use a mediator (e.g. mail, dedicated server etc.),
once established, the network can assume the position of mediator and start
working autonomously.

Random peer sampling protocols provide each editor with a local neighborhood
table. These tables allow communicating to a subset of peers (here
editors). Messages can travel from neighbor to neighbor to reach any editor in a
scalable manner.

%% Web browsers exist in most of current devices. \TODO{Moar.}

%% \SPRAY is an adaptive random peer sampling protocol which provides each editor
%% with a neighborhood table the size of which grows and shrinks following the
%% global network size, yet without any global knowledge. 

\SPRAY's lifecycle at each editor comprises three steps that aim to provide and
maintain logarithmically scaling neighborhood tables compared to the editing
session size:
\begin{asparadesc}
\item [\textbf{The joining:}] the editor wants to join the editing session. It
  contacts an editor inside the network and waits for its answer to establish
  the very first WebRTC connection between them. Using this connection, the
  joining editor asks to its contact editor to operate as mediator to advertise
  its presence to the contact's neighbors. Each of these neighbors connects to
  the joining peer. Since the number of neighbor is supposed logarithmic, the
  number of connections in the network grows by $\log(N)+1$ where $\log(N)$ is
  the natural logarithm of the network size $N$. Following such growth, the
  number of connections globally is $N.\log(N)$.
\item [\textbf{The shuffling:}] each editor regularly initiates a table
  shuffling with the oldest of its neighbors. The initiator serves as mediator
  between half of its neighborhood chosen at random and the oldest neighbor. In
  response, the oldest neighbor does the same. As a result, they roughly have an
  identical neighborhood size which tackles the load-balancing problem that may
  rise from the joining part of the protocol. It is worth noting that the number
  of connection remains constant. In addition, the references in neighborhood
  tables are randomly distributed. The network converges in exponential time
  toward a topology exposing properties similar to those of random
  graphs~\cite{erdos1959random}.
\item [\textbf{The leaving:}] the editor can leave the network without giving
  notice, i.e., equivalent to a crash. Nevertheless, its removal from the system
  must lead to a decrease of $\log(N)+1$ connections while currently it
  represents $\log(N)$ connections from its neighborhood tables, and $\log(N)$
  connections from other editors' neighborhood tables pointing to it. The
  adjusting falls under the responsibility of these other editors. For this
  purpose, they detect the departure of a neighbor when they try to initiate a
  shuffle with it. Since such detection presumably happen $\log(N)$ times, they
  all duplicate one of their neighbors except for one; the probability being
  processed using their own neighborhood table size. 
\end{asparadesc}

Such random peer sampling protocol constitutes the ground of various protocols
such as topology managements. \CRATE uses it to disseminate all operations of
the replicated sequence to all collaborators in a scalable manner. Referred as
epidemic dissemination, rumor mongering, or gossiping, the protocol is a
two-step operation:
\begin{inparaenum}[(i)]
\item Each operation performed on the local document replica leads to the
  creation of a message. The latter is sent to all the neighborhood provided by
  \SPRAY.
\item Each editor receiving such broadcast message transmits it again to its
  neighborhood.
\end{inparaenum}
Thus, messages transitively reach all collaborators very efficiently.  This
dissemination protocol inherits from \SPRAY's adaptiveness. Hence, traffic
follows the editing session size.

Prior approaches generally oversize the partial views to handle largest groups
of users. Unfortunately, smaller groups suffer from such \emph{a priori}
dimensioning. Since \SPRAY automatically adjusts neighborhood table sizes, it
equally handles from smallest to largest groups.

%%% Local Variables:
%%% mode: latex
%%% TeX-master: "../paper"
%%% End:
