
\begin{abstract}
  Real-time collaborative editors are common tools to distribute work
  across space, time, and organizations. Unfortunately, current
  mainstream editors such as Google Docs rely on central servers which
  require ressources and raises privacy and scalability issues.
  \CRATE is a real-time decentralized collaborative editor that runs
  directly in web browsers thanks to WebRTC. Compared to state of art,
  \CRATE is the first real-time editor that only require browsers to
  support collaboration, raises no privacy issue and handle
  transparently small groups to large groups of users. Consequently,
  \CRATE is not restricted to traditional collaborative editing but
  can also be used in massive online lectures, TV shows or large
  conferences to allow users to share their notes. \CRATE properties
  rely on two scientific results: i) a replicated sequence structure
  with sub-linear upper-bound space complexity; this prevents the
  editor to run costly distributed garbage collectors, ii) an adaptive
  peer-sampling protocol; this prevent the editor to oversize routing
  tables and let small networks pay the price of large networks. This
  paper describes \CRATE and its properties. It proposes a live
  experiment where \CRATE will be deployed on the large group of
  WWW2016 participants. We aim to collect a real-world editing corpus
  to validate results obtained in lab.
\end{abstract}

\keywords{Collaborative editor, decentralized, real-time}

%%% Local Variables:
%%% mode: latex
%%% TeX-master: "../paper"
%%% End:
