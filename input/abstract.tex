
\begin{abstract}
  Real-time collaborative editors are common tools to distribute work across
  space, time, and organizations. Unfortunately, mainstream editors such as
  Google Docs rely on central servers and raises privacy and scalability issues.
  \CRATE is a real-time decentralized collaborative editor that runs directly in
  web browsers thanks to WebRTC. Compared to state-of-the-art, \CRATE is the
  first real-time editor that only requires browsers to support collaborative
  editing and to transparently handle from small to large groups of
  users. Consequently, \CRATE can also be used in massive online lectures, TV
  shows or large conferences to allow users to share their notes. \CRATE
  properties rely on two scientific results:
  \begin{inparaenum}[(i)]
  \item a replicated sequence structure with sub-linear upper bound on space
    complexity; this prevents the editor from running costly distributed garbage
    collectors,
  \item an adaptive peer sampling protocol; this prevent the editor from
    oversizing routing tables, hence from letting small networks pay the price
    of large networks.
  \end{inparaenum}
  This paper describes \CRATE, its properties and its usage.
\end{abstract}

\keywords{Collaborative editor, decentralized, real-time}

%%% Local Variables:
%%% mode: latex
%%% TeX-master: "../paper"
%%% End:
