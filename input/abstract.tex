
\begin{abstract}
  Real-time collaborative editors are common tools to distribute work
  across space, time, and organizations. Unfortunately, mainstream
  editors such as Google Docs rely on central servers and raises
  privacy and scalability issues.  \CRATE is a real-time decentralized
  collaborative editor that runs directly in web browsers thanks to
  WebRTC. Compared to state of art, \CRATE is the first real-time
  editor that only require browsers to support collaborative editing
  and handle transparently small groups to large groups of
  users. Consequently, \CRATE can also be used in massive online
  lectures, TV shows or large conferences to allow users to share
  their notes. \CRATE properties rely on two scientific results: i) a
  replicated sequence structure with sub-linear upper-bound space
  complexity; this prevents the editor to run costly distributed
  garbage collectors, ii) an adaptive peer-sampling protocol; this
  prevent the editor to oversize routing tables and let small networks
  pay the price of large networks. This paper describes \CRATE, its
  properties and its usage.
\end{abstract}

\keywords{Collaborative editor, decentralized, real-time}

%%% Local Variables:
%%% mode: latex
%%% TeX-master: "../paper"
%%% End:
